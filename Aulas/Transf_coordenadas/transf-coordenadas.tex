\documentclass[10pt,a4paper]{article}
\usepackage[utf8]{inputenc}
\usepackage[portuguese]{babel}
\usepackage[T1]{fontenc}
\usepackage{amsmath}
\usepackage{amsfonts}
\usepackage{amssymb}
\author{Vanderlei C. Oliveira Jr.}
\begin{document}

\begin{center}
\begin{large}
\textbf{Esquema iterativo para transformar coordenadas Cartesianas 
geocêntricas $(x,y,z)$ em coordenadas geodésicas $(h,\phi,\lambda)$
referidas ao elipsoide WGS84}
\end{large}
\end{center}

\bigskip
\bigskip

\underline{Coordenadas Cartesianas geocêntricas}

\begin{itemize}
\item $x = X_{0} \, m$ 
\item $y = Y_{0} \, m$
\item $z = Z_{0} \, m$
\end{itemize}

\bigskip

\underline{Elipsoide de refer\^{e}ncia WGS84}:

\begin{itemize}
\item Semi-eixo maior: $a = 6378137.0 \, m$
\item Achatamento: $f = 1/298.257223563$
\item Semi-eixo menor: $b = a \, (1 - f) \, m$
\item $1^{a}$ excentricidade: $e = \dfrac{\sqrt{a^{2} - b^{2}}}{a}$
\end{itemize}

\bigskip

\underline{Esquema iterativo}

\begin{enumerate}

\item $p = \sqrt{X_{0}^{2} + Y_{0}^{2}}$

\item $\phi_{1} = tan^{-1} \left( \dfrac{Z_{0}}{p (1 - e^{2})} 
                           \right)$

\item $N_{1} = \dfrac{a^{2}}{\sqrt{\left( a \, cos\phi_{1} \right)^{2} + 
                                   \left( b \, sen\phi_{1} \right)^{2}}}$

\item $h_{1} = \dfrac{p}{cos\phi_{1}} - N_{1}$

\item $h_{0} = h_{1}$, $\phi_{0} = \phi_{1}$ e $N_{0} = N_{1}$

\item $i = 0$ e $ITMAX = 5$ (por exemplo)

\item Enquanto $i < ITMAX$

\begin{enumerate}

\item $\phi_{1} = tan^{-1} \left( \dfrac{Z_{0}}{p} 
                           \left( 1 - \dfrac{e^{2} N_{1}}{N_{1} + h_{1}} 
                           \right)^{-1}
                           \right)$
\item Etapa 3

\item Etapa 4

\item Etapa 5

\item $i \leftarrow i + 1$

\end{enumerate}

\item $\lambda = tan^{-1} \left( \dfrac{Y_{0}}{X_{0}} \right)$

\end{enumerate}

\bigskip

\underline{Observaç\~{o}es}

\begin{itemize}

\item Os \^{a}ngulos devem estar em radiano

\item Para testar a sua implementaç\~{a}o:

\begin{itemize}

\item Defina coordenadas geod\'{e}sicas
      $h = H$, $\phi = \Phi$ e $\lambda = \Lambda$

\item Calcule as coordenadas Cartesianas $x = X_{0}$, 
      $y = Y_{0}$ e $z = Z_{0}$

\item Utilize o esquema iterativo descrito acima para
      estimar o valor das coordenadas geod\'{e}sicas 
      $H$, $\Phi$ e $\Lambda$ que voc\^{e} definiu
      previamente


\end{itemize}

\end{itemize}

\end{document}